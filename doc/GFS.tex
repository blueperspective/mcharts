%% LyX 1.6.2 created this file.  For more info, see http://www.lyx.org/.
%% Do not edit unless you really know what you are doing.
\documentclass[english]{article}
\usepackage[T1]{fontenc}
\usepackage[latin9]{inputenc}


%%%%%%%%%%%%%%%%%%%%%%%%%%%%%% LyX specific LaTeX commands.
%% Because html converters don't know tabularnewline
\providecommand{\tabularnewline}{\\}

\usepackage{babel}

\begin{document}

\title{GrisbiCharts : General Functional Specifications}

\maketitle

\part{Core}


\section{Accounting objects}


\subsection{Data manipulation}


\subsubsection{Criteria}

A criteria is an object to select a subset of data. A criteria is
composed of base criteria and compound criteria.

A base criteria is about period or category names.

A compound criteria contains one or more criteria : the negation criteria
inverse a given criteria ; the and list criteria makes a logical and
with a list of criteria.


\subsubsection{Aggregation}

Aggregation is the fact

Identified aggregations

\begin{tabular}{|c|c|c|}
\hline 
Id & Name & Description\tabularnewline
\hline
\hline 
 & AmountAggregator & \tabularnewline
\hline 
 & AmountPeriodAggregator & \tabularnewline
\hline
\end{tabular}

Accounting matrix

The accounting matrix is the future of the aggregation, with a generic
aggregation object. 

For now, you have to select either the AmountAggregtor (1d) or the
AmountPeriodAggregator (2d), and applied restrictions (see ) depends
on the type of aggregator. 

The aim of the accounting matrix is to have a multi-dimentional array
of values.

The following table shows how an accounting matrix is representated:

\begin{tabular}{|c|c|c|}
\hline 
Period & Category & Value\tabularnewline
\hline
\hline 
01/01/09 & food & 150.00\tabularnewline
\hline 
01/01/09 & car & 200.00\tabularnewline
\hline 
01/01/09 & house & 600.00\tabularnewline
\hline 
01/02/09 & food & 160.00\tabularnewline
\hline 
01/02/09 & car & 250.00\tabularnewline
\hline 
01/02/09 & house & 550.00\tabularnewline
\hline
\end{tabular}


\subsubsection{Restriction}

The restriction is applied to an aggregation and is similar to a criteria
in the idea : it is used to select only a subset of the aggregated
values.

Identified restrictions

\begin{tabular}{|c|c|c|}
\hline 
Id & Name & Description\tabularnewline
\hline
\hline 
 & NTopRestriction & Restricts the elements to the top N firsts.\tabularnewline
\hline
\end{tabular}


\part{Ui}
\end{document}
